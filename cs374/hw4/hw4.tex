% ---------
%  Compile with "pdflatex hw0".
% --------
%!TEX TS-program = pdflatex
%!TEX encoding = UTF-8 Unicode

\documentclass[11pt]{article}
\usepackage{jeffe,handout,graphicx}
\usepackage[utf8]{inputenc}		% Allow some non-ASCII Unicode in source
\usepackage{amsmath}
\usepackage[makeroom]{cancel}

%  Redefine suits
\usepackage{pifont}
\def\Spade{\text{\ding{171}}}
\def\Heart{\text{\textcolor{Red}{\ding{170}}}}
\def\Diamond{\text{\textcolor{Red}{\ding{169}}}}
\def\Club{\text{\ding{168}}}

\def\Cdot{\mathbin{\text{\normalfont \textbullet}}}
\def\Sym#1{\textbf{\texttt{\color{BrickRed}#1}}}



% =====================================================
%   Define common stuff for solution headers
% =====================================================
\Class{CS/ECE 374}
\Semester{Fall 2018}
\Authors{1}
\AuthorOne{Will Koster}{jameswk2@illinois.edu}
%\Section{}

% =====================================================
\begin{document}

% ---------------------------------------------------------


\HomeworkHeader{4}{1}	% homework number, problem number

\begin{quote}
    A {\bf two-dimensional} Turing machine (2D TM for short) uses an infinite
  two-dimensional grid of cells as the tape. For simplicity assume
  that the tape cells corresponds to integers $(i,j)$ with $i,j \ge 0$;
  in other words the tape corresponds to the positive quadrant of the
  two dimensional plane. The machine crashes if it tries to move below
  the $x=0$ line or to the left of the $y=0$ line.  The transition
  function of such a machine has the form
  $\delta: Q \times \Gamma \rightarrow Q \times \Gamma \times
  \{L,R,U,D,S\}$ where $L$, $R$, $U$, $D$ stand for ``left'',
  ``right'', ``up'' and ``down'' respectively, and $S$ stands for
  ``stay put''. You can assume that the input to the 2D TM is written
  on the first row and that its head is initially at location $(0,0)$.
  Argue that a 2D TM can be simulated by an
  ordinary TM (1D TM); it may help you to use a multi-tape TM for
  simulation. In particular address the following points.
  \begin{itemize}
  \item How does your TM store the grid cells of a 2D TM on a one dimensional
    tape?
  \item How does your TM keep track of the head position of the
    2D TM?
  \item How does your 1D TM simulate one step of the
    2D TM?
  \end{itemize}
  If a 2D TM takes $t$ steps on some input how many steps
  (asymptotically) does your simulating 1D TM take on the same input?
  Give an asymptotic estimate. Note that it is quite difficult to give
  a formal proof of the simulation argument, hence we are looking for
  high-level arguments similar to those we gave in lecture for various
  simulations.

\end{quote}
\hrule



\begin{solution}
    The grid of the 2D TM will be arranged on the 1D tape with the rows written one after the other in order separated by some separator symbol. The position of the head will be denoted by a separate set of symbols such that $\gamma'$ is the symbol $\gamma$ when the head is positioned over it. Each step, the head will start at the current position of the 2D TM, and the state of the 2D TM will be encoded in the state of our 1D TM. It will read the $\delta$ function of the 2D TM and write the necessary symbol. If the direction is left or right, it will move one left or one right. If moving right puts it on a separator, it will put a blank space there and shift everything else down. Moving up and down means counting how many spaces are between the head and the previous separator and going that many spaces after the next separator. This might mean inserting some blanks, just as before. Then it marks the new space as the present space, moves to a new state, and starts over. \\

    Since the $t$th step can take at most $kt^2$ steps, since there can be at most $ct$ slots simulated on the tape after $t$ steps and it might have to insert $t$ characters, each taking $t$ time. This means that $t$ steps will run in $O(t^3)$.
\end{solution}

\HomeworkHeader{4}{2}	% homework number, problem number

\begin{quote}
    Suppose you are given $k$ sorted arrays $A_1,A_2,\ldots,A_k$
  each of which has $n$ numbers. Assume that all numbers in the arrays
  are distinct. You would like to merge them into single sorted array
  $A$ of $kn$ elements. Recall that you can merge two sorted arrays of
  sizes $n_1$ and $n_2$ into a sorted array in $O(n_1+n_2)$ time.
  \begin{itemize}
  \item Use a divide and conquer strategy to merge the sorted arrays
    in $O(nk \log k)$ time. To prove the correctness of the algorithm
    you can assume a routine to merge two sorted arrays.
  \item In MergeSort we split the array of size $N$ into two arrays
    each of size $N/2$, recursively sort them and merge the two sorted
    arrays. Suppose we instead split the array of size $N$ into $k$
    arrays of size $N/k$ each and use the merging algorithm in the
    preceding step to combine them into a sorted array.  Describe the
    algorithm formally and analyze its running time via a recurrence.
    You do not need to prove the correctness of the recursive algorithm.
  \end{itemize}
\end{quote}
\hrule



\begin{solution}
    \begin{enumerate}
        \item Our algorithm goes as follows. \\
            MergeN([$A_1, A_2 \ldots A_k$]) \\
            if n < 5 then merge by brute force and return \\
            left := MergeN([$A_1, A_2 \ldots A_{\frac{k}{2}}$]) \\
            right := MergeN([$A_{\frac{k}{2} + 1}, \ldots A_k$]) \\
            merged := merge(left, right) \\
            return merged \\
            \\
            The time recurrence for this is 
            \[
                T(k < 5, n) = O(1) 
            \]
            \[
                T(k, n) = 2T\left(\frac{k}{2}, n\right) + kn
            \]
            I claim that this recurrence has a closed form of $nk \log k$. \\
            Let's prove by induction on $k$, number of arrays to merge. \\
            First, we know by the definition of the recurrence that this runs in constant time for any $k < 5$. \\
            Lets take some $k \geq 6$ and assume that for any $p < k$, $T(p, n) = O(nk \log k)$ for any $n$.
            \[
                T(k, n) = 2T\left(\frac{k}{2}, n\right) + kn
            \]
            \[
                T(k, n) = 2\left(O\left(n\frac{k}{2} \log \frac{k}{2}\right)\right) + kn
            \]
            \[
                T(k, n) = O\left(nk \log \frac{k}{2}\right) + kn
            \]
            \[
                T(k, n) = O\left(nk \log k\right) + kn
            \]
            \[
                T(k, n) = O\left(nk \log k\right)
            \]
            QED \\
            \\
            Now we must prove that our algorithm is correct. Again, we will prove by induction on $k$, the number of arrays to merge. \\
            Strictly speaking, we are proving that, given $k$ sorted arrays of length $n$, our algorithm returns one sorted array of size $nk$ containing all the elements of the original $k$ arrays. \\
            For any $k < 5$, our algorithm uses brute force, so it works trivially. \\
            For any $k \geq 6$, assume that we know for any $p < k$. We know by our inductive hypothesis that the arrays named \textit{left} and \textit{right} are sorted and of length $\frac{nk}{2}$. \\
            The merge function was given, so we know it works, and so we know our merged list has the right values and they are sorted. \\
            QED
        \item Our new merge sort works as follows: \\
            MergeKSort(A[1,2,3, \ldots, n], k): \\
            if n < 5 sort using brute force and return \\
            Form lists $L_1, L_2, \ldots, L_k$ where $L_i = [A[\frac{(i - 1)n}{k} + 1],\ldots,A[\frac{in}{k}]]$ \\
            return MergeN([MergeKSort($L_1$, k), MergeKSort($L_2$, k), MergeKSort($L_3$, k), \ldots, MergeKSort($L_k$, k)]) \\ 
            The time recurrence is as follows: 
            \[
                T(n < 5, k) = O(1)
            \]
            \[
                T(n, k) = kT(\frac{n}{k}, k) + O(nk \log k)
            \]
            At each level of the recursion tree, $O(nk\log k)$ work will be done, and there will be $\log_kn$ levels, so our closed form is 
            \[
                T(n, k) = O((\log_kn)(nk \log k))
            \]

    \end{enumerate}
\end{solution}

\HomeworkHeader{4}{3}	% homework number, problem number

\begin{quote}
    It is common these days to hear statistics about wealth inequality in
the United States. A typical statement is that the the top 1\% of
earners together make more than ten times the total income of the bottom
70\% of earners. You want to verify these statements on some data
sets. Suppose you are given the income of people as an $n$ element
\emph{unsorted} array $A$, where $A[i]$ gives the income of person
$i$.
\begin{itemize}
\item Describe an $O(n)$-time algorithm that given $A$ checks whether
  the top 1\% of earners together make more than ten times the bottom
  70\% together. Assume for simplicity that $n$ is a multiple of 100
  and that all numbers in $A$ are distinct. Note that sorting $A$ will
  easily solve the problem but will take $\Omega(n\log n)$ time.
\item More generally we may want to compute the total earnings of the
  top $\alpha$\% of earners for various values of $\alpha$. Suppose we
  are given $A$ and $k$ numbers $\alpha_1 < \alpha_2 < \ldots < \alpha_k$
  each of which is a number between $0$ and $100$ and we wish to
  compute the total earnings of the top $\alpha_i$\% of earners for
  each $1 \le i \le k$. Assume for simplicity that $\alpha_i n$ is an
  integer for each $i$. Describe an algorithm for this problem that
  runs in $O(n \log k)$ time. Note that sorting will allow you to
  solve the problem in $O(n \log n)$ time but when $k \ll n$, $O(n
  \log k)$ is faster. Note that an $O(nk)$ time algorithm is relative
  easy. {\em Hint:} Use the previous part with $\alpha_{k/2}$ first and then
  use divide and conquer.
\end{itemize}

You should prove the correctness of the second part of the problem.
It helps to write a recursive algorithm so that you can use induction
to prove correctness.

\end{quote}
\hrule



\begin{solution}
    \begin{enumerate}
        \item First, let's define a helper function, the idea for which I stole from lecture (if not the exact implementation). Also, the select function is the same QuickSelect function we derived in lecture. \\
            TopK(A[1,2,3 \ldots n], k, d $\in$ \{top, bottom\}):\\
            Form lists $L_1, L_2, \ldots, L_{\ceil{n / 5}}$ where $L_i = \{A[5i - 4],\ldots,A[5i]\}$. \\
            B = $[ b_i | b_i \text{is the median of }L_i \text{ found by brute force}]$ \\ 
            b = select(B, $\ceil{n / 10}$) \\
            Partition $A$ into $A_\text{less}$ and $A_\text{greater}$ using $b$ as pivot \\
            if d = top then j = n - k \\
            else j = k \\
            if $|A_\text{less}|$ = j and d = top then return [b] + $A_\text{greater}$ \\ 
            else if $|A_\text{less}|$ = j and d = bottom then return $A_\text{less}$ + [b] \\ 
            else if $|A_\text{less}|$ < j and d = bottom then return $A_\text{less}$ + [b] + TopK($A_\text{greater}$, j - $|A_\text{less}|$, bottom) \\
            else if $|A_\text{less}|$ > j and d = bottom then return TopK($A_\text{less}$, j, bottom) \\
            else if $|A_\text{less}|$ < j and d = top then return TopK($A_\text{greater}$, k, top) \\
            else if $|A_\text{less}|$ > j and d = top then return $A_\text{greater}$ + [b] + TopK($A_\text{less}$, j - 1 - $|A_\text{greater}|$, top) \\

            Now that we have that, the actual function is actually pretty simple. \\
            OccupyWallSt(A[1,2,3 \ldots, n]): \\
            topOne := TopK(A, $\frac{99n}{100}$, top) \\
            bottomSeventy := TopK(A, $\frac{70n}{100}$, bottom) \\
            return sum(topOne) / sum(bottomSeventy) > 10

        \item Our function to find the earnings of the top $a\%$ for each $a$ in $as$ looks like this: \\
            TopAs(A[1,\ldots,n], as[1,2,3,\ldots,k]): \\
            if k < 5 compute by calling TopK k times (naive algorithm) \\
            leftHalf = TopK(A, n(as[$\frac{k}{2}$]), bottom) \\
            rightHalf = TopK(A, n(as[$\frac{k}{2}$]), top) \\
            return TopAs(leftHalf, as[1,2,\ldots,$\frac{k}{2} - 1$]) + [sum(rightHalf)] + TopAs(rightHalf, as[$\frac{k}{2}$ + 1],\ldots,k]) \\

            Let's prove that this works. Specifically, we are proving that, if we call our function TopAs with a list of incomes $A$ and a list of percentages $as$, we will get a list $L$ such that $L_i$ is the total earnings of the top $as[i]$ percent of earners. \\
            We will prove using induction on $k$, the number of percentages. \\
            First, for the case where $k < 5$, we know it works trivially. 
            For the case where $k \geq 5$, assume that we know for any $p < k$, TopAs works properly. \\
            We can assume TopK works, so the leftHalf and rightHalf variables are correct. Namely, leftHalf is a list of the bottom $as[\frac{k}{2}]\%$ of incomes and rightHalf is a list of the top $as[\frac{k}{2}]\%$. \\
            By our inductive hypothesis, we know that TopAs(leftHalf, as[1,2,\ldots,$\frac{k}{2} - 1$]) returns $L_i$ for $i \leftarrow [1\ldots \frac{k}{2} - 1]$ and TopAs(rightHalf, as[$\frac{k}{2}$ + 1],\ldots,k]) returns $L_i$ for $i \leftarrow [\frac{k}{2} + 1\ldots k]$. \\
            Since we know that rightHalf is the list of all the incomes of the top $as[\frac{k}{2}]\%$ of earners, $sum(as[\frac{k}{2}]\%)$ is their total income, $L_\frac{k}{2}$. Thus, we return $L$. \\
            QED \\

            The time recurrence is 
            \[
                T(n, k < 5) = kn
            \]
            This is because we are just calling the $O(n)$ time TopK algorithn $k$ times.
            \[
                T(n, k) = O(n) + T\left(pn, \frac{k}{2}\right) + T\left((1 - p)n, \frac{k}{2}\right)
            \]
            where $p$ is some number between 0 and 1. We don't know how much of $A$ the left subcall will get compared to the right, but we know that together they will see everything in $A$. \\

            We can prove that this runs in $O(n\log k)$ using induction on $k$, the length of the $as$ array. \\
            We know for $k < 5$, $T(n, k) = O(n\log k)$ by the definition of our recurrence. \\
            Assume that we know, for any $p < k$, $T(n, p) = O(n\log k)$. 
            \[
                T(n, k) = O(n) + T\left(pn, \frac{k}{2}\right) + T\left((1 - p)n, \frac{k}{2}\right)
            \]
            By our inductive hypothesis, 
            \[
                T(n, k) = O(n) + O\left(pn\log\frac{k}{2}\right) + O\left((1 - p)n\log\frac{k}{2}\right)
            \]
            This simplifies to 
            \[
                T(n, k) = O\left(n\log k\right)
            \]

    \end{enumerate}
\end{solution}
\end{document}
