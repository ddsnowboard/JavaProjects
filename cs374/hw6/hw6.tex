% ---------
%  Compile with "pdflatex hw0".
% --------
%!TEX TS-program = pdflatex
%!TEX encoding = UTF-8 Unicode

\documentclass[11pt]{article}
\usepackage{jeffe,handout,graphicx}
\usepackage[utf8]{inputenc}		% Allow some non-ASCII Unicode in source
\usepackage{amsmath}
\usepackage[makeroom]{cancel}

%  Redefine suits
\usepackage{pifont}
\def\Spade{\text{\ding{171}}}
\def\Heart{\text{\textcolor{Red}{\ding{170}}}}
\def\Diamond{\text{\textcolor{Red}{\ding{169}}}}
\def\Club{\text{\ding{168}}}

\def\Cdot{\mathbin{\text{\normalfont \textbullet}}}
\def\Sym#1{\textbf{\texttt{\color{BrickRed}#1}}}

\newcommand{\IsSinL}{\text{IsStringInL}}
\newcommand{\IsSinlang}[1]{\text{IsStringIn}L_{#1}}
\newcommand{\cost}{\text{cost}}



% =====================================================
%   Define common stuff for solution headers
% =====================================================
\Class{CS/ECE 374}
\Semester{Fall 2018}
\Authors{1}
\AuthorOne{Will Koster}{jameswk2@illinois.edu}
%\Section{}

% =====================================================
\begin{document}

% ---------------------------------------------------------


\HomeworkHeader{6}{1}	% homework number, problem number

\begin{quote}
The McKing chain wants to open several restaurants along Red
  street in Shampoo-Banana. The possible locations are at $L_1,L_2,
  \ldots, L_n$ where $L_i$ is at distance $m_i$ meters from the start
  of Red street. Assume that the street is a straight line and the
  locations are in increasing order of distance from the starting
  point (thus $0 \leq m_1 < m_2 < \ldots < m_n$). McKing has collected
  some data indicating that opening a restaurant at location $L_i$
  will yield a profit of $p_i$ independent of where the other
  restaurants are located. However, the city of Shampoo-Banana has a
  zoning law which requires that any two McKing locations should be
  $D$ or more meters apart. {\em In addition McKing does not want to
    open more than $k$ restaurants due to budget constraints.}
  Describe an algorithm that McKing can use to figure out the maximum
  profit it can obtain by opening restaurants while satisfying the
  city's zoning law and the constraint of opening at most $k$
  restaurants.  Your algorithm should use only $O(n)$ space and you
  should not assume that $k$ is a constant.
\end{quote}
\hrule



\begin{solution}
    This basically works off the assumption that you should do the thing with the most gross profit (ie, profit minus opportunity cost) at any time. If you only get k restaurants, do that k times, and you'll be golden. I think this works, but I'm not 100\% sure. 
    sum = 0
    k.times do
        Get array of gross profit (p_i - opportunity cost)
        remove highest gross profit j and everyone it disallowed from l
        sum += p_j
    return sum
    kn^2
    The above is totally wrong. If you have d = 1, k = 1, and three slots in a row 30-35-30, it doesn't work. I thought about increasing the opportunity cost when k = 1 to reflect the fact that you're paying for the opportunity to have any others (ie, the opportunity cost of each store is the profit all the other possibilities on the street, not just your neighbors), but I think there's a situation where that doesn't work. I can't think of it, but the fact that you're not amortizing the cost of depleting k can't be good. I don't think this works but I keep coming back to it for some reason. Maybe I'll learn something. So I thought that the real opportunity cost of putting a restaurant somewhere isn't the sum of all the neighbors, it's only the sum of the highest one, since the exact cost of putting it there includes depleting k and for that same depletion of k, you only can place 1. That won't work because you could have the case where you have 3 restaurants (m, p) = (0, 19), (10, 30), (20, 20), k = 2 and d = 10. I thought you could pick the top k neighbors, but that won't work for that example either. 
    I did figure out that if you look at this as a graph where vertices are locations and E = (u, v) st |u - v| < d, then you have a maximum weight independent set problem where the size of the independent set is k (which is a pretty rare problem, apparently) and your graph is chordal, which means no cycles greater than 3. This has all kinds of properties, and I'm sure some can help me, but I can't figure out which. 
    I don't know how to encode the finite size into our algorithm. Maybe I do it as if k is infinite and then in a separate loop iterate up to k? That would be O(n + k), and k is not constant, but it is less than n, so that still linear? That could be the ticket. I like this idea.
\end{solution}

\HomeworkHeader{6}{2}	% homework number, problem number

\begin{quote}
Let $X = x_1,x_2,\ldots,x_r$, $Y = y_1,y_2,\ldots,y_s$ and $Z =
  z_1,z_2,\ldots,z_t$ be three sequences. A common {\em supersequence}
  of $X$, $Y$ and $Z$ is another sequence $W$ such that $X$, $Y$ and $Z$
  are subsequences of $W$. Suppose $X = a,b,d,c$ and $Y = b,a,b,e,d$ and $Z =
  b, e, d, c$. A simple common supersequence of $X$, $Y$ and $Z$ is
  the concatenation of $X$, $Y$ and $Z$ which is
  $a,b,d,c,b,a,b,e,d,b,e,d,c$ and has length $13$. A shorter one is
  $b, a, b, e, d, c$ which has length $6$.  Describe an efficient
  algorithm to compute the {\em length} of the shortest common
  supersequence of three given sequences $X$, $Y$ and $Z$. You may want to
  first solve the two sequence problem to get you strated.
\end{quote}
\hrule



\begin{solution}
    sheep
\end{solution}

\HomeworkHeader{6}{3}	% homework number, problem number

\begin{quote}
Given a graph $G=(V,E)$ a matching is a subset of edges in $G$
  that do not \emph{intersect}. More formally $M \subseteq E$ is a
  matching if every vertex $v \in V$ is incident to at most one edge
  in $M$. Matchings are of fundamental importance in combinatorial
  optimization and have many applications. Given $G$ and non-negative
  weights $w(e), e \in E$ on the edges one can find the maximum weight
  matching in a graph in polynomial time but the algorithm requires
  advanced machinery and is beyond the scope of this course. However,
  finding the maximum weight matching in a tree is easier via dynamic
  programming. 
\end{quote}
\hrule



\begin{solution}
    goats
\end{solution}
\end{document}
