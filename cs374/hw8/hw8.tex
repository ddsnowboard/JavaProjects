% ---------
%  Compile with "pdflatex hw0".
% --------
%!TEX TS-program = pdflatex
%!TEX encoding = UTF-8 Unicode

\documentclass[11pt]{article}
\usepackage{jeffe,handout,graphicx}
\usepackage[utf8]{inputenc}		% Allow some non-ASCII Unicode in source
\usepackage{amsmath}
\usepackage[makeroom]{cancel}

%  Redefine suits
\usepackage{pifont}
\def\Spade{\text{\ding{171}}}
\def\Heart{\text{\textcolor{Red}{\ding{170}}}}
\def\Diamond{\text{\textcolor{Red}{\ding{169}}}}
\def\Club{\text{\ding{168}}}

\def\Cdot{\mathbin{\text{\normalfont \textbullet}}}
\def\Sym#1{\textbf{\texttt{\color{BrickRed}#1}}}


% =====================================================
%   Define common stuff for solution headers
% =====================================================
\Class{CS/ECE 374}
\Semester{Fall 2018}
\Authors{1}
\AuthorOne{Will Koster}{jameswk2@illinois.edu}
%\Section{}

% =====================================================
\begin{document}

% ---------------------------------------------------------


\HomeworkHeader{8}{1}	% homework number, problem number

\begin{quote}
Graphs are a powerful tool to model many phenomena.  The edges
  of a graph model pairwise relationships. It is natural to consider
  higher order relationships.  Indeed {\em hypergraphs} provide one
  such modeling tool.  A hypergraph $G=(V,\mathcal{E})$ consists of a
  finite set of nodes/vertices $V$ and a finite set of hyper-edges
  $\mathcal{E}$. A hyperedge $e \in \mathcal{E}$ is simply a subset of
  nodes and the cardinality of the subset can be larger than two. An
  undirected graph is a hypergraph where each hyper-edge is of size
  exactly two. Here is an example. $V=\{1,2,3,4,5\}$ and $\mathcal{E}
  = \{\{1,2\},\{2,3,4\},\{1,3,4,5\},\{2,5\}\}$. The representation
  size of a hypergraph is $|V| + \sum_{e \in \mathcal{E}} |e|$.  An
  alternating sequence of nodes and edges
  $x_1,e_1,x_2,e_2,\ldots,e_{k-1},x_k$ where $x_i \in V$ for $1 \le i
  \le k$ and $e_j \in \mathcal{E}$ for $1 \le j \le k-1$ is called a
  path from $u$ to $v$ if (i) $x_1 = u$ and $x_k = v$ and (ii) for $1
  \le j < k$, $x_j \in e_j$ and $x_{j+1} \in e_j$.

  \begin{itemize}
  \item Given a hypergraph $G=(V,\mathcal{E})$ and two nodes $u,v \in V$
    we say that $u$ is connected to $v$ if there is path from $u$ to $v$.
    We say that a hypergraph is connected
    if each pair of nodes $u,v$ in $G$ are connected. Describe an algorithm that
    given a hypergraph $G$ checks whether $G$ is connected in linear time.
    In essense describe a reduction of this problem to the standard graph
    connectivity problem. You need to prove the correctness of your algorithm.
  \item Suppose we want to quickly spread a message from one person to
    another person during an emergency on a social network called
    AppsWhat which is organized as a collection of groups. Messages
    sent by a group member are broadcast to the entire group. AppsWhat
    knows the members and the list of groups on its service. The goal
    is to find the fewest messages that need to be sent such that a
    person $u$ can reach a person $v$. Model this problem using
    hypergraphs and describe a linear-time algorithm for it.
    No proof necessary for this part.
  \end{itemize}
\end{quote}
\hrule



\begin{solution}
    ants
\end{solution}

\HomeworkHeader{8}{2}	% homework number, problem number

\begin{quote}
Let $G=(V,E)$ be a directed graph with edge lengths that can be
  negative. Let $\ell(e)$ denote the length of edge $e \in E$ and
  assume it is an integer. Assume you have a shortest path tree $T$
  rooted at a source node $s$ that contains all the nodes in $V$. You
  also have the distance values $d(s,u)$ for each $u \in V$ in an
  array (thus, you can access the distance from $s$ to $u$ in $O(1)$
  time). Note that the existence of $T$ implies that $G$ does not have
  a negative length cycle. 
  \begin{itemize}
  \item  Let $e=(p,q)$ be an edge of $G$ that is {\em not} in
    $T$. Show how to compute in $O(1)$ time the smallest
    integer amount by which we can decrease $\ell(e)$ before $T$
    is not a valid shortest path tree in $G$. Briefly justify the correctness of your solution.
  \item Let $e=(p,q)$ be an edge in the tree $T$. Show how to
  compute in $O(m+n)$ time the smallest integer amount by which we can 
  increase $\ell(e)$ such that $T$ is no longer a valid shortest path tree. 
  Your algorithm should output $\infty$ if no amount of increase will
  change the shortest path tree. Briefly justify the correctness of your solution.
  \end{itemize}
  The example below may help you. The dotted 
 edges form the shortest path tree $T$ and the distances to the nodes
  from $s$ are shown inside the circles. For the first part consider
  an edge such as $(b,d)$ and for the second part consider an edge
  such as $(f,e)$.
\begin{figure}[h]
  \centering
  \includegraphics[height=2.5in]{example}
\end{figure}
\end{quote}
\hrule



\begin{solution}
    beetles
\end{solution}

\HomeworkHeader{8}{3}	% homework number, problem number

\begin{quote}
Since you are taking an algorithms class you decided to create a
  fun candy hunting game for Halloween. You set up a maze with one
  way streets that can be thought of as a directed graph
  $G=(V,E)$. Each node $v$ in the maze has $w(v)$ amount of candy
  located at $v$.
  \begin{itemize}
  \item Each of your friends, starting at a given node $s$, has to
    figure out the maximum amount of candy they can collect. Note that
    candy at node $v$ can be collected only once even if the node $v$
    is visited again on the way to some other place.
  \item Your friends complain that they can collect more candy if they
    get to choose the starting node. You agree to their their request
    and ask them to maximize the amount of candy they can collect
    starting at any node they choose.
  \end{itemize}
  Before you ask your friends to solve the game you need to know how
  to do it yourself!  Describe efficient algorithms for both variants.
  Ideally your algorithm should run in linear time.
  {\em Hint:} Consider what happens if $G$ is strongly connected and
  if it is a DAG.

  No proof necessary if you use reductions to standard algorithms via
  graph transformations and simple steps. Otherwise you need to prove
  the correctness.
\end{quote}
\hrule



\begin{solution}
    Rocks
\end{solution}
\end{document}
