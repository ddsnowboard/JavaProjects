% ---------
%  Compile with "pdflatex hw0".
% --------
%!TEX TS-program = pdflatex
%!TEX encoding = UTF-8 Unicode

\documentclass[11pt]{article}
\usepackage{jeffe,handout,graphicx}
\usepackage[utf8]{inputenc}		% Allow some non-ASCII Unicode in source
\usepackage{amsmath}
\usepackage[makeroom]{cancel}

%  Redefine suits
\usepackage{pifont}
\def\Spade{\text{\ding{171}}}
\def\Heart{\text{\textcolor{Red}{\ding{170}}}}
\def\Diamond{\text{\textcolor{Red}{\ding{169}}}}
\def\Club{\text{\ding{168}}}

\def\Cdot{\mathbin{\text{\normalfont \textbullet}}}
\def\Sym#1{\textbf{\texttt{\color{BrickRed}#1}}}


% =====================================================
%   Define common stuff for solution headers
% =====================================================
\Class{CS/ECE 374}
\Semester{Fall 2018}
\Authors{1}
\AuthorOne{Will Koster}{jameswk2@illinois.edu}
%\Section{}

% =====================================================
\begin{document}

% ---------------------------------------------------------


\HomeworkHeader{8}{1}	% homework number, problem number

\begin{quote}
Graphs are a powerful tool to model many phenomena.  The edges
  of a graph model pairwise relationships. It is natural to consider
  higher order relationships.  Indeed {\em hypergraphs} provide one
  such modeling tool.  A hypergraph $G=(V,\mathcal{E})$ consists of a
  finite set of nodes/vertices $V$ and a finite set of hyper-edges
  $\mathcal{E}$. A hyperedge $e \in \mathcal{E}$ is simply a subset of
  nodes and the cardinality of the subset can be larger than two. An
  undirected graph is a hypergraph where each hyper-edge is of size
  exactly two. Here is an example. $V=\{1,2,3,4,5\}$ and $\mathcal{E}
  = \{\{1,2\},\{2,3,4\},\{1,3,4,5\},\{2,5\}\}$. The representation
  size of a hypergraph is $|V| + \sum_{e \in \mathcal{E}} |e|$.  An
  alternating sequence of nodes and edges
  $x_1,e_1,x_2,e_2,\ldots,e_{k-1},x_k$ where $x_i \in V$ for $1 \le i
  \le k$ and $e_j \in \mathcal{E}$ for $1 \le j \le k-1$ is called a
  path from $u$ to $v$ if (i) $x_1 = u$ and $x_k = v$ and (ii) for $1
  \le j < k$, $x_j \in e_j$ and $x_{j+1} \in e_j$.

  \begin{itemize}
  \item Given a hypergraph $G=(V,\mathcal{E})$ and two nodes $u,v \in V$
    we say that $u$ is connected to $v$ if there is path from $u$ to $v$.
    We say that a hypergraph is connected
    if each pair of nodes $u,v$ in $G$ are connected. Describe an algorithm that
    given a hypergraph $G$ checks whether $G$ is connected in linear time.
    In essense describe a reduction of this problem to the standard graph
    connectivity problem. You need to prove the correctness of your algorithm.
  \item Suppose we want to quickly spread a message from one person to
    another person during an emergency on a social network called
    AppsWhat which is organized as a collection of groups. Messages
    sent by a group member are broadcast to the entire group. AppsWhat
    knows the members and the list of groups on its service. The goal
    is to find the fewest messages that need to be sent such that a
    person $u$ can reach a person $v$. Model this problem using
    hypergraphs and describe a linear-time algorithm for it.
    No proof necessary for this part.
  \end{itemize}
\end{quote}
\hrule



\begin{solution}
    \begin{itemize}
        \item I stole this from the Wikipedia article on hypergraphs. There was no further citation. 
            \\ Making the new vertex set $V' = V \cup \mathcal{E}$ in linear time is obvious. Here's how we'll make our new edge set $E'$.
            \begin{algo}
                $E' = \emptyset$
                \\ for e in $\mathcal{E}$ \+
                \\ for v in e \+
                \\ $E' = E' \cup \{(v, e)\}$ 
            \end{algo}
            Our new regular undirected graph is $G' = (V', E')$.
            \\ This runs in $|V| + |\mathcal{E}| + \sum_{e \in \mathcal{E}} |e| = O(|V| +\sum_{e \in \mathcal{E}}|e|)$. Then we just run some search to see whether $G'$ it is connected, which we know takes linear time.
            \\ Now let's prove that if some graph $G'$ so constructed is connected, then the original hypergraph $G$ is also connected. Note that this graph $G'$ is bipartite, with the two partitions being hyperedges and vertices. Obviously, no hyperedge is adjacent to any other hyperedge, nor is any vertex adjacent to any other vertex. Since it's connected, we know that for every pair of distinct vertices $(u, v) \mid u, v \in V$, there is some path going between $u$ and $v$. Note that we can say this for a bunch of other pairs too, but we don't care about them. Since $G'$ is bipartite, any path in $G'$ alternates between edges and vertices of $G$. We also know by the definition of $E'$ that there is an edge $(u \in V, v \in \mathcal{E})$ iff $u \in v$. Therefore, we know that for some path $x_1, e_1, x_2, e_2,\ldots, e_k, x_k$, for any $j$ from $1$ to $k - 1$ inclusive, $x_j \in e_j$ and $x_{j+1} \in e_j$.
            \\ This fits the definition of a path in a hypergraph. Therefore, we know that there is a path between any two vertices $u, v \in V$ in $G$, so $G$ is connected.
        \item We can model this situation as a hypergraph by making groups into hyperedges and individuals into vertices. You have a vertex for every user of the services, and every hyperedge contains all the vertices that are in the corresponding group. Here's what it looks like mathematically. 
            \[
                V = \{\text{All the people using AppWhats}\}
            \]
            \[
                \mathcal{E} = \{\{\text{All the people in group $g$}\} \text{for each group $g$ on the service}\}
            \]
            The answer to this problem would be the length of the shortest path between $u$ and $v$. We can use the definition of a path from above and we can say that the length of some path $P$ is the count of elements $e \in P$ such that $e \in V$.
            \\ To implement this, we just transform the hypergraph to a regular graph as above and then run BFS from $u$ until we find $v$. If we keep track of each node's parent in the BFS tree, we can just climb up from $v$ to $u$ and count all the vertices $v \in V$ we find along the way. 
    \end{itemize}
\end{solution}

\HomeworkHeader{8}{2}	% homework number, problem number

\begin{quote}
Let $G=(V,E)$ be a directed graph with edge lengths that can be
  negative. Let $\ell(e)$ denote the length of edge $e \in E$ and
  assume it is an integer. Assume you have a shortest path tree $T$
  rooted at a source node $s$ that contains all the nodes in $V$. You
  also have the distance values $d(s,u)$ for each $u \in V$ in an
  array (thus, you can access the distance from $s$ to $u$ in $O(1)$
  time). Note that the existence of $T$ implies that $G$ does not have
  a negative length cycle. 
  \begin{itemize}
  \item  Let $e=(p,q)$ be an edge of $G$ that is {\em not} in
    $T$. Show how to compute in $O(1)$ time the smallest
    integer amount by which we can decrease $\ell(e)$ before $T$
    is not a valid shortest path tree in $G$. Briefly justify the correctness of your solution.
  \item Let $e=(p,q)$ be an edge in the tree $T$. Show how to
  compute in $O(m+n)$ time the smallest integer amount by which we can 
  increase $\ell(e)$ such that $T$ is no longer a valid shortest path tree. 
  Your algorithm should output $\infty$ if no amount of increase will
  change the shortest path tree. Briefly justify the correctness of your solution.
  \end{itemize}
  The example below may help you. The dotted 
 edges form the shortest path tree $T$ and the distances to the nodes
  from $s$ are shown inside the circles. For the first part consider
  an edge such as $(b,d)$ and for the second part consider an edge
  such as $(f,e)$.
\end{quote}
\hrule



\begin{solution}
    \begin{itemize}
        \item Just to avoid any confusion, our edge $e = (p, q)$ starts at $p$ and goes to $q$. The real goal here is to decrease $\ell(e)$ until it becomes the shortest path from $p$ to $q$. We can find the distance expended getting from $p$ to $q$; it's just $d(s, q) - d(s, p)$. If we make $\ell(e) < d(s, q) - d(s, p)$, then the path from $p$ to $q$ in $T$ will no longer be the shortest path, but rather taking $e$ directly will, or it could create a negative-length cycle. Thus, decrementing $\ell(e)$ by $-(d(s, q) - d(s, p) - 1 - \ell(e))$ will invalidate your tree.
        \item If I do a DFS on T, I can find the discovery times for every node. Using those, I can easily find whether one depends on another. If I find edges that go from a node whose weight is independent of $\ell(e)$ to one whose weight depends on $\ell(e)$, then I know that that edge could become part of T if I increase $\ell(e)$ enough (or, failing that, it could show some shorter path than T somewhere else). 
            \begin{algo}
                Depends(A, B) \+
                \\ // Does A depend on B (ie, if B's distance increases, will A's increase too?)
                \\ // Note that every node depends on itself
                \\ return A.startTime $\leq$ B.startTime \&\& B.endTime $\leq$ A.endTime \-

                \\ SmallestIncrease($e = (p, q)$) \+
                \\ Explore T using DFS, placing startTime and endTime on each node 
                \\ changes = $\{\infty\}$
                \\ for each edge $(a, b)$ where $(a, b) \not \in T$ and Depends($p, a$) and Depends($b, q$) \+
                \\ changes = changes $\cup \{d(s, p) + \ell((a, b)) - d(s, m) + 1\}$ \-
                \\ smallestChange = min(changes)
                \\ return smallestChange
            \end{algo}
            If I increase the weight of $e$ such that the distance from some vertex $a$ to another $b$ is greater than some extant edge not in $T$ $(a, b)$, we are guaranteed that $T$ is no longer valid, since $(a, b)$ is shorter than the path in $T$ going from $a$ to $b$.
            \\ I know DFS takes $O(|V| + |E|)$ time, and I'm doing a constant amount of work for each edge, so that part of the algorithm takes $O(|E|)$ time, so the whole algorithm takes $O(|V|+|E|)$ time. 
    \end{itemize}
\end{solution}

\HomeworkHeader{8}{3}	% homework number, problem number

\begin{quote}
Since you are taking an algorithms class you decided to create a
  fun candy hunting game for Halloween. You set up a maze with one
  way streets that can be thought of as a directed graph
  $G=(V,E)$. Each node $v$ in the maze has $w(v)$ amount of candy
  located at $v$.
  \begin{itemize}
  \item Each of your friends, starting at a given node $s$, has to
    figure out the maximum amount of candy they can collect. Note that
    candy at node $v$ can be collected only once even if the node $v$
    is visited again on the way to some other place.
  \item Your friends complain that they can collect more candy if they
    get to choose the starting node. You agree to their their request
    and ask them to maximize the amount of candy they can collect
    starting at any node they choose.
  \end{itemize}
  Before you ask your friends to solve the game you need to know how
  to do it yourself!  Describe efficient algorithms for both variants.
  Ideally your algorithm should run in linear time.
  {\em Hint:} Consider what happens if $G$ is strongly connected and
  if it is a DAG.

  No proof necessary if you use reductions to standard algorithms via
  graph transformations and simple steps. Otherwise you need to prove
  the correctness.
\end{quote}
\hrule



\begin{solution}
    If you have a strongly-connected graph, this is trivial; since you can get to any vertex from any other, you can get to every house in your neighborhood no matter where you start. If you have a DAG, you can use DP to find the maximum by saying that the maximum candy available in any subtree is the candy available at the root added to the most candy available by going through any of the root's children. 
    \\ \\ If you run the linear-time SCC algorithm, you have a DAG of strongly connected graphs. From here this is pretty simple: for every node of the meta-graph $v$, you say $w(v) = \text{sum}(w(v') \mid v' \in v)$. Make a topological sort $L[1..n]$ on the meta-graph. For each $L[i]$ for $i$ from $n\ldots 1$, set $L[i].max = w(L[i]) + max(\{w(p) \mid \text{$p$ is a child of $L[i]$}\})$. The answer will be $s.max$. 
    \\ For part 2, do the same thing as above, except instead of returning $s.max$, find the node $n$ such that $n.max$ is maximized.

    \\ We know making the SCC meta-graph takes linear time. Getting the weights of all the strongly-connected components takes linear time since each vertex in the original graph is counted only once. Topological sorting is just DFS, which takes linear time. Calculating $L[i].max$ for every node takes linear time because each node is child to at most one other node, so we see each node a constant number of times. Finally, looping through all the vertices to get the answer for part 2 is obviously linear time. Therefore, this whole algorithm takes linear time.
\end{solution}
\end{document}
