% ---------
%  Compile with "pdflatex hw0".
% --------
%!TEX TS-program = pdflatex
%!TEX encoding = UTF-8 Unicode

\documentclass[11pt]{article}
\usepackage{jeffe,handout,graphicx}
\usepackage[utf8]{inputenc}		% Allow some non-ASCII Unicode in source
\usepackage{amsmath}
\usepackage[makeroom]{cancel}

%  Redefine suits
\usepackage{pifont}
\def\Spade{\text{\ding{171}}}
\def\Heart{\text{\textcolor{Red}{\ding{170}}}}
\def\Diamond{\text{\textcolor{Red}{\ding{169}}}}
\def\Club{\text{\ding{168}}}

\def\Cdot{\mathbin{\text{\normalfont \textbullet}}}
\def\Sym#1{\textbf{\texttt{\color{BrickRed}#1}}}



% =====================================================
%   Define common stuff for solution headers
% =====================================================
\Class{CS/ECE 374}
\Semester{Fall 2018}
\Authors{1}
\AuthorOne{Will Koster}{jameswk2@illinois.edu}
%\Section{}

% =====================================================
\begin{document}

% ---------------------------------------------------------


\HomeworkHeader{1}{1}	% homework number, problem number

\begin{quote}
    For each of the following languages over the alphabet ${0,1}$, give a regular expression that describes that language, and briefly argue why your expression is correct.
    \begin{enumerate}
        \item All strings except $\Sym{010}$.
        \item All strings that end in $\Sym{10}$ and contain $\Sym{101}$ as a substring.
        \item All strings in which every nonempty maximal substring of $\Sym{1}$s is of length divisible by 3. For instance, $\Sym{0110}$ and $\Sym{101110}$ are not in the language while $\Sym{11101111110}$ is.
        \item All strings that do not contain the substring $\Sym{010}$.
        \item All strings that do not contain the \textit{subsequence} $\Sym{010}$.
    \end{enumerate}
\end{quote}
\hrule



\begin{solution}
    \begin{enumerate}
        \item $\epsilon + 0 + 00(1+0)^* + 1(1+0)^* + 01 + 011(1+0)^* + 010(1+0)^+$ \\
            This is probably not the most efficient way to write this regular expression, but if you draw the DFA for the language containing only $\Sym{010}$ and then invert it, following that DFA trivially (if tediously) yields this regular expression.
        \item $(1+0)^*(101(1+0)^*10 + 1010)$ \\
            The obvious case is some string with a bunch of characters, then a $\Sym{101}$, then a bunch more characters, and then $\Sym{10}$. That is captured in the first arm of the $+$. The sneaky part is that the shortest string in this language is only 4 characters long, $\Sym{1010}$, since the $\Sym{101}$ and the $\Sym{10}$ are not separate. The second arm of the $+$ accounts for that. 
        \item $(0^*(111)^* 0^*)^*$ \\
            Each group has some number of zeros, since they don't really matter, and then some integral number of the group $\Sym{111}$, which means that the number of ones in that run will be divisible by 3, then more zeros, and then it starts over. Each group of ones will have length divisible by 3, which is the whole definition of the language.
        \item $(1+0^+11)^*0^+(1+\epsilon)$ \\ 
            I made the DFA for the compliment of this language and then inverted it. Strings that match the first parenthesis group any amount of times are the same as the empty string with respect to the DFA. Then I just got all the strings that could get to one of the other accepting states, which weren't that complicated.
        \item $0^*(\epsilon + 1^*(\epsilon + 01^*))$ \\
            The string can start with any number of zeros (including none) and then finish. Incipient zeros don't really matter, so once they're gone, we canhave any number of ones, but once we've seen a one, we can have no more than 1 zero in the rest of the string, although we can have as many more ones as we want.
    \end{enumerate}
\end{solution}

\HomeworkHeader{1}{2}	% homework number, problem number

\begin{quote}
    Let $L$ be the set of all strings in $\{\Sym{0}, \Sym{1}\}^*$ that contain an even number of $\Sym{0}$s and an odd number of $\Sym{1}$s and does not contain the substring $\Sym{01}$.\\
        Describe a DFA over the alphabet $\Sigma = \{\Sym{0}, \Sym{1}\}$ that accepts the language $L$. Argue that your machine accepts every string in $L$ and nothing else by explaining what each state in your DFA \textit{means}. You should be able to create one with five states.
\end{quote}
\hrule



\begin{solution}
    First, $Q$ = \{EvenOnes, OddOnes, Failed, PositiveEvenZeros, OddZeros\} \\
    $s$ = EvenOnes \\
    $A$ = \{PositiveEvenZeros, OddOnes\} \\
    $\delta(\text{EvenOnes}, 1) = \text{OddOnes}$ \\
    $\delta(\text{EvenOnes}, 0) = \text{Failed}$ \\
    $\delta(\text{OddOnes}, 1) = \text{EvenOnes}$ \\
    $\delta(\text{OddOnes}, 0) = \text{OddZeros}$ \\
    $\delta(\text{OddZeros}, 0) = \text{PositiveEvenZeros}$ \\
    $\delta(\text{PositiveEvenZeros}, 0) = \text{OddZeros}$ \\
    $\delta(\text{PositiveEvenZeros}, 1) = \text{Failed}$ \\
    $\delta(\text{OddZeros}, 1) = \text{Failed}$ \\
    $\delta(\text{Failed}, 1) = \text{Failed}$ \\
    $\delta(\text{Failed}, 0) = \text{Failed}$ \\
    The meaning of the states should be clear from the names. The only unclear parts are that both *Ones states imply zero (ie, an even number) of zeros, beacause seeing a one from any *Zeros state means that we've seen the substring $\Sym{01}$, so we fail immediately, so being in a *Ones state means we've never seen a zero before.
\end{solution}

\HomeworkHeader{1}{3}	% homework number, problem number

\begin{quote}
    Let $L_1, L_2,L_3$ and $L_4$ be regular languages over $\Sigma$
  accepted by DFAs $M_1 = (Q_1, \Sigma, \delta_1, s_1, A_1)$,
  $M_2 = (Q_2, \Sigma, \delta_2, s_1, A_2)$, $M_3 = (Q_3,
  \Sigma, \delta_3, s_3, A_3)$, and $M_4 = (Q_4,
  \Sigma, \delta_4, s_4, A_4)$ respectively.

\begin{enumerate}
\item Describe a DFA $M = (Q, \Sigma, \delta, s, A)$ in terms of $M_1,
  M_2,M_3$ and $M_4$ that accepts $L = (L_1 - L_2) \cap (L_4 \cup \overline{L_3})$.  Formally specify the components $Q, \delta, s,$ and $A$
  for $M$ in terms of the components of $M_1, M_2,M_3, M_4$.
\item Prove by induction that your construction is correct.
\end{enumerate}
\end{quote}
\hrule



\begin{solution}
    \begin{enumerate}
        \item $Q$ = $Q_1 \times Q_2 \times Q_3 \times Q_4$ \\
            $\delta((s1,s2,s3,s4),t) = (\delta_1(s1,t),\delta_2(s2,t),\delta_3(s3,t),\delta_4(s4,t))$ \\
            $s = (s_1,s_2,s_3,s_4)$ \\
            $A = \{(s1, s2, s3, s4) \textbf{ : } s1 \in A_1, s2 \cancel{\in} A_2, \text{ and either } s3 \cancel{\in} A_3\text{ or } s4 \in A_4 \}$
        \item Prove that some string $w$ is accepted by our machine (ie, $\delta^*(s, w) \in A$) $M$ iff it is in $L$ (ie, in $L_1$ and not $L_2$ and in $L_4$ or the compliment of $L_3$, or both). \\
            Prove by induction on the length of $w$.\\
            There are two possibilities: 
            \begin{enumerate}
                \item $w = \epsilon$ \\
                    By the definition of the $\delta$ function for DFAs, the final state for this string is the same as $s$, namely $(s_1, s_2, s_3, s_4)$. By the definition of $A$, this $s$ state is in $A$ iff it is in $A_1$, not in $A_2$, and in one or both of $A_4$ and the complement of $A_3$. By the definitions of $A_n$ for n from 1 to 4 and the definition of $L$, this means that $w$ is in $L$.
                \item $w = ax$ \\
                    First, assume that we know that $p \in L \iff p$ is accepted by $M$ for all strings $p$ shorter than $w$.
            \end{enumerate}
    \end{enumerate}
\end{solution}
\end{document}
