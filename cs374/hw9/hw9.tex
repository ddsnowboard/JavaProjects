% ---------
%  Compile with "pdflatex hw0".
% --------
%!TEX TS-program = pdflatex
%!TEX encoding = UTF-8 Unicode

\documentclass[11pt]{article}
\usepackage{jeffe,handout,graphicx}
\usepackage[utf8]{inputenc}		% Allow some non-ASCII Unicode in source
\usepackage{amsmath}
\usepackage[makeroom]{cancel}

%  Redefine suits
\usepackage{pifont}
\def\Spade{\text{\ding{171}}}
\def\Heart{\text{\textcolor{Red}{\ding{170}}}}
\def\Diamond{\text{\textcolor{Red}{\ding{169}}}}
\def\Club{\text{\ding{168}}}

\def\Cdot{\mathbin{\text{\normalfont \textbullet}}}
\def\Sym#1{\textbf{\texttt{\color{BrickRed}#1}}}


% =====================================================
%   Define common stuff for solution headers
% =====================================================
\Class{CS/ECE 374}
\Semester{Fall 2018}
\Authors{1}
\AuthorOne{Will Koster}{jameswk2@illinois.edu}
%\Section{}

% =====================================================
\begin{document}

% ---------------------------------------------------------


\HomeworkHeader{9}{1}	% homework number, problem number

\begin{quote}
Spanning trees have many nice algorithmic properties and are useful
  in a number of applications. For those interested, see the connection
  to abstract structures called matroids.
  \begin{itemize}
  \item Consider the following ``local-search'' algorithm for MST.  It
    starts with an arbitrary spanning tree $T$ of $G$. Suppose
    $e=(u,v)$ is an edge in $G$ that is not in $T$. It checks if it
    can add $e$ to $T$ and remove an edge $e'$ on the unique path
    $p_T(u,v)$ from $u$ to $v$ in $T$ such that tree $T' = T - e' + e$
    is cheaper than $T$.  If $T'$ is cheaper then it replaces $T$ by
    $T'$ and repeats. Assuming all edge weights are integers one can
    see that the algorithm will terminate with a ``local-optimum'' $T$
    which means it cannot be improved further by these single-edge
    ``swaps''.  Assuming all edge weghts are distinct prove that a
    local-optimum tree is an MST. Note that you are not concerned with
    the running time here.
  \item We saw in lecture that Borouvka's algorithm for MST can be
  implemented in $O(m \log n)$ time where $m$ is the number of edges
  and $n$ is the number of nodes. We also saw that Prim's algorithm
  can be implemented in $O(m + n \log n)$ time.
  Obtain an algorithm
    for MST with running time $O(m \log \log n)$ by running Borouvka's
    algorithm for some number of steps and then switching to Prim's
    algorithm. This algorithm is better than either of the algorithms
    when $m = \Theta(n)$. Formalize the algorithm, specify the
    parameters and argue carefully about the implementation and running
    time details. No proof of correctness required but your algorithm should be
    clear.
  \end{itemize}
\end{quote}
\hrule



\begin{solution}
    \begin{itemize}
        \item Assume the contrary. That is, assume that we have some local-optimum tree $T$ that is not an MST. 
            \\ Since all edge weights are distinct, we know there exists exactly one MST $M$ for this graph. 
            From our given assumption, it differs from $T$ in some way. Since we know all spanning trees have $|V| - 1$ edges, 
            we know that there is some edge $e_1$ that appears only in $T$. If we make a new graph $M' = M \cup \{e_1\}$, we know there
            will be a cycle since $M$ is a spanning tree. If there is some other edge $e_2$ in the cycle that has smaller weight 
            than $e_1$, then we have a contradiction, since our local search algorithm would have found this edge and never would have
            given us $T$. If $e_1$ is the smallest edge, then we have another contradiction, since $e_1$ would have to be in any MST, 
            but it isn't in $M$. Since we have a contradiction in either case, we prove the contrary, that any such local-optimum tree is an MST.
    \end{itemize}
\end{solution}

\HomeworkHeader{9}{2}	% homework number, problem number

\begin{quote}
Suppose you have just purchased a new type of hybrid car that
  uses fuel extremely efficiently, but can only travel $100$ miles on a
  single battery. The car’s fuel is stored in a single-use battery,
  which must be replaced after at most $100$ miles. The actual fuel is
  virtually free, but the batteries are expensive and can only be
  installed by licensed battery-replacement technicians. Thus, even if
  you decide to replace your battery early, you must still pay full
  price for the new battery to be installed. Moreover, because these
  batteries are in high demand, no one can afford to own more than one
  battery at a time.  Suppose you are trying to get from San Francisco
  to New York City on the new InterContinental Super-Highway, which
  runs in a direct line between these two cities. There are several
  fueling stations along the way; each station charges a different
  price for installing a new battery. Before you start your trip, you
  carefully print the Wikipedia page listing the locations and prices
  of every fueling station on the ICSH.

  Given this information, how do
  you decide the best places to stop for fuel?  More formally, suppose
  you are given two arrays $D[1 .. n]$ and $C[1 .. n]$, where $D[i]$ is the
  distance from the start of the highway to the ith station, and $C[i]$
  is the cost to replace your battery at the $i$th station. Assume that
  your trip starts and ends at fueling stations (so $D[1] = 0$ and $D[n]$
  is the total length of your trip), and that your car starts with an
  empty battery (so you must install a new battery at station $1$).


  \begin{itemize}
  \item   Describe and analyze a greedy algorithm to find the minimum number
  of refueling stops needed to complete your trip. Don’t forget to
  prove that your algorithm is correct.
\item But what you really want
  to minimize is the total cost of travel. Show that your greedy
  algorithm in the preceding part does not produce an optimal solution when
  extended to this setting.
  \end{itemize}
\end{quote}
\hrule



\begin{solution}
test
\end{solution}

\HomeworkHeader{9}{3}	% homework number, problem number

\begin{quote}
Define a TM generator (or enumerator) $G$ as a TM with a single
  work tape, but also a special write-only output tape that starts
  with $\#$ at the left most cell, and the write head one cell to the
  right of that.  Assume for simplicity that the input alphabet is
  binary.  From time to time, depending on its computation on the work
  tape, the generator may write a character on the output tape and
  move the output head to the right.  It is said to generate a word
  $w$ if at some point in time, $\#w\#$ is on the output tape. The
  language generated by $G$, denoted by $L(G)$, is the set of all
  strings that $G$ ever generates.
  \begin{itemize}
  \item A generator $G$ is well-behaved if the strings it generates
    are lexicographically ordered. Prove that a language $L$ is
    recursive if and only if there is a well-behaved generator $G$ such
    that $L=L(G)$. Note that you need
    to show both directions separately. {\em Hint:} Treat the case of finite and
    infinite languages separately.

  \item Prove that a language $L$ is recursively enumerable if and
    only if $L = L(G)$ for some TM generator $G$. Note that you need
    to show both directions separately. {\em Hint:} Use the idea of
    dovetailing.
  \end{itemize}
  In proving the statement it suffices to show a construction. We
  describe one case. Suppose you want to show that $L$ is recursively
  enumerable implies that there is a generator $G$ such that
  $L = L(G)$. In order to prove this you need to come up with an
  algorithm that uses, as its input, a description/code
  $\langle M \rangle$ of a TM $M$ and outputs the description/code
  $\langle G \rangle$ of generator TM $G$ such that $L(G) = L(M)$.  You
  do not have to actually prove the correctness of your algorithm as long as
  the description is clear.
\end{quote}
\hrule



\begin{solution}
test
\end{solution}
\end{document}
