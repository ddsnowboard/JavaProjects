% ---------
%  Compile with "pdflatex hw0".
% --------
%!TEX TS-program = pdflatex
%!TEX encoding = UTF-8 Unicode

\documentclass[11pt]{article}
\usepackage{jeffe,handout,graphicx}
\usepackage[utf8]{inputenc}		% Allow some non-ASCII Unicode in source
\usepackage{amsmath}
\usepackage[makeroom]{cancel}

%  Redefine suits
\usepackage{pifont}
\def\Spade{\text{\ding{171}}}
\def\Heart{\text{\textcolor{Red}{\ding{170}}}}
\def\Diamond{\text{\textcolor{Red}{\ding{169}}}}
\def\Club{\text{\ding{168}}}

\def\Cdot{\mathbin{\text{\normalfont \textbullet}}}
\def\Sym#1{\textbf{\texttt{\color{BrickRed}#1}}}

\newcommand{\IsSinL}{\text{IsStringInL}}
\newcommand{\IsSinlang}[1]{\text{IsStringIn}L_{#1}}
\newcommand{\cost}{\text{cost}}



% =====================================================
%   Define common stuff for solution headers
% =====================================================
\Class{CS/ECE 374}
\Semester{Fall 2018}
\Authors{1}
\AuthorOne{Will Koster}{jameswk2@illinois.edu}
%\Section{}

% =====================================================
\begin{document}

% ---------------------------------------------------------


\HomeworkHeader{5}{1}	% homework number, problem number

\begin{quote}
    Given a graph $G=(V,E)$ a vertex cover of $G$ is a subset
  $S \subseteq V$ of vertices such that for every edge $(u,v) \in E$,
  $u$ or $v$ is in $S$. The goal in the minimum vertex cover problem
  is to find a vertex cover $S$ of smallest size. In the weighted
  version of the problem, vertices have non-negative weights
  $w: V \rightarrow \mathbb{Z}_+$, and the goal is to find a vertex
  cover of minimum weight.  You can find some examples and discussion
  at the following Wikipedia link
  \url{https://en.wikipedia.org/wiki/Vertex_cover}.  Describe a {\em
    recursive} algorithm that given a graph $G=(V,E)$ and weights
  $w(v), v \in V$ outputs a vertex cover of $G$ with minimum
  weight. Do not worry about the running time.

\end{quote}
\hrule



\begin{solution}
    The intiution for this is that we are just going to try every subset of $V$. This will take quite a long time (specifically, $2^n$ in the number of vertices) but apparently that's OK.
    \begin{algo}
        MinVertexCover((V, E), w) \+
        \\ lowestWeight = $\infty$
        \\ smallestCover = V
        \\ if E = $\emptyset$ \+
        \\      return 0 \-
        \\ for vertex in V \+
        \\      v$'$ = V $\setminus$ \{vertex\}
        \\      e$'$ = $\{ ed = (u,v) | ed \in E, u \neq vertex \text{ and } v \neq vertex\}$
        \\      cover = \{vertex\} + MinVertexCover((v$'$, e$'$), w)
        \\      weight = sum($\{ w(v) | v \in \text{cover}\}$)
        \\          if weight < lowestWeight \+
        \\               lowestWeight = weight 
        \\               smallestCover = cover \- \-
        \\ return lowestWeight
    \end{algo}
\end{solution}

\HomeworkHeader{5}{2}	% homework number, problem number

\begin{quote}
    Let $\Sigma$ be a finite alphabet and let $L_1$ and $L_2$ be two
  languages over $\Sigma$. Assume you have access to two routines
  $\IsSinlang{1}(u)$ and $\IsSinlang{2}(u)$. The former routine
  decides whether a given string $u$ is in $L_1$ and the latter
  whether $u$ is in $L_2$. Using these routines as black boxes
  describe an efficient algorithm that given an arbitrary string
  $w \in \Sigma^*$ decides whether $w \in (L_1 \cup L_2)^*$. To
  evaluate the running time of your solution you can assume that calls
  to $\IsSinlang{1}()$ and $\IsSinlang{2}()$ take constant time. Note
  that you are not assuming any property of $L_1$ or $L_2$ other than
  being able to test membership in those languages.
\end{quote}
\hrule



\begin{solution}
    Imagine a string in $(L_1\cup L_2)^*$ as $w_1w_2w_3\ldots w_n$, where each $w_i \in (L_1\cup L_2)$. The recursive algorithm would try to find every way to split $w$ into smaller strings until it finds one such that every smaller string is in $(L_1\cup L_2)$. My algorithm is just going to find all the ways to split off the first substring until it finds one such that the first substring is in $(L_1\cup L_2)$ and the rest of the string has already been confirmed in $(L_1\cup L_2)^*$. $arr$ is an array such that $arr[i] = \text{true}$ iff $w$ from the $i$th character to the end is in $(L_1\cup L_2)^*$.
    \\ Just for clarity, I'm going to use Python-style list indices/ranges. Also arrays are going to be zero-indexed. 
    \begin{algo}
        IsInBoth(w) \+
        \\ arr = [false] * $|w|$ + [true] // This is just a list of $|w|$ false values and 1 true value.
        \\ // The last one is a sentinel signifying an empty string, since 
        \\ // $\epsilon$ is in $L^*$ for any $L$.
        \\ for start in range($|w|$)[::-1] // This just iterates over the string indices backwards\+
        \\ for end in range(start + 1, $|w| + 1$) \+
        \\ if ($\IsSinlang{1}$(w[start:end]) or $\IsSinlang{2}$(w[start:end])) and arr[end] \+
        \\ arr[start] = true
        \\ break \- \- \-
        \\ return arr[0]
    \end{algo}
    Since we go through some portion of the whole string for each letter in the string, this runs in $O(n^2)$ time. 
\end{solution}

\HomeworkHeader{4}{3}	% homework number, problem number

\begin{quote}
    Recall that a \emph{palindrome} is any string that is exactly
  the same as its reversal, like \Sym{I}, or \Sym{DEED}, or
  \Sym{RACECAR}, or \Sym{AMANAPLANACATACANALPANAMA}. For technical
  reasons, in this problem, we will only be interested in palindromes
  that are of length at least one, hence we will
  not treat the string $\epsilon$ as a palindrome.

Any string can be decomposed into a sequence of palindrome substrings.  For example, the string \Sym{BUBBASEESABANANA} (“Bubba sees a banana.”\@) can be broken into palindromes in the following ways (among many others):
\begin{gather*}
	\Sym{BUB} \Cdot \Sym{BASEESAB} \Cdot \Sym{ANANA}\\
	\Sym{B} \Cdot \Sym{U} \Cdot \Sym{BB} \Cdot \Sym{A} \Cdot \Sym{SEES} \Cdot \Sym{ABA} \Cdot \Sym{NAN} \Cdot \Sym{A}\\
	\Sym{B} \Cdot \Sym{U} \Cdot \Sym{BB} \Cdot \Sym{A} \Cdot \Sym{SEES} \Cdot \Sym{A} \Cdot \Sym{B} \Cdot \Sym{ANANA}\\
	\Sym{B} \Cdot \Sym{U} \Cdot \Sym{B} \Cdot \Sym{B} \Cdot
        \Sym{A} \Cdot \Sym{S} \Cdot \Sym{E} \Cdot \Sym{E} \Cdot
        \Sym{S} \Cdot \Sym{A} \Cdot \Sym{B} \Cdot \Sym{ANA}  \Cdot
        \Sym{N} \Cdot \Sym{A}\\
        	\Sym{B} \Cdot \Sym{U} \Cdot \Sym{B} \Cdot \Sym{B}
                \Cdot \Sym{A} \Cdot \Sym{S} \Cdot \Sym{E} \Cdot
                \Sym{E} \Cdot \Sym{S} \Cdot \Sym{A} \Cdot \Sym{B}
                \Cdot \Sym{A} \Cdot \Sym{N} \Cdot \Sym{A}  \Cdot \Sym{N} \Cdot \Sym{A}
\end{gather*}

Since any string $w \neq \epsilon$ can always be decomposed to palindromes we
may want to find a decomposition that optimizes some objective. Here
are two example objectives. The first objective is to decompose $w$ into the
smallest number of palindromes. A second objective is to find a
decomposition such that each palindrome in the decomposition has
length at least $k$ where $k$ is some given input parameter. Both of
these can be cast as special cases of an abstract problem. Suppose we
are given a function called $cost()$ that takes a positive integer $h$
as input and outputs an integer $\cost(h)$.  Given a
decomposition of $w$ into $u_1,u_2,\ldots, u_r$ (that is,
$w = u_1u_2\ldots u_r$) we can define the cost of the decomposition as
$\sum_{i=1}^r \cost(|u_i|)$.

For example if we define $\cost(h) = 1$ for all $h \ge 1$ then finding
a minimum cost palindromic decomposition of a given string $w$ is the
same as finding a decomposition of $w$ with as few palindromes as
possible. Suppose we define $\cost()$ as follows: $\cost(h) = 1$
for $h < k$ and $\cost(h) = 0$ for $h \ge k$. Then finding a
minimum cost palindromic decomposition would enable us to decide
whether there is a decomposition in which all palindromes are
of length at least $k$; it is possible iff the minimum cost is $0$.

Describe an efficient algorithm that given  black box access to
a function $\cost()$, and a string $w$, outputs the
value of a minimum cost palindromic decomposition of $w$.
\end{quote}
\hrule



\begin{solution}
    First, let's find a way to tell whether a substring is a palindrome. For some string $w$, define $w(n, m)$ be the string containing the $n$th through $m$th characters in $w$, inclusive. So if $w = \text{ILLINOIS}$, $w(3, 5) = \text{LIN}$, and $w(1, |w|) = w$ for any $w$.
    We can define a function $IsPalindrome(n, m)$ to tell us whether $w(n, m)$ is a palindrome for some string $w$. 
\[
    \emph{IsPalindrome}(n,m) =
	\begin{cases}
		\textsc{True} & \text{if $n=m$}
\\[0.5ex]
                w[n] = w[m] & \text{if $n = m - 1$}
\\[0.5ex]
        w[n] = w[m] \wedge \emph{IsPalindrome}(n + 1, m - 1) & \text{otherwise}
\\[0.5ex]
	\end{cases}
\]
We can memoize this by using a 2-d array indexed from 1 to $n$. Since each cell depends on at most the cell 1 up on the $n$ axis and 1 down on the $m$ axis, if done in the right order, they can each be calculated in $O(1)$ time, allowing the whole structure to be created in $O(n^2)$ time. The for-loops showing the correct order would look like this:
\begin{algo}
    for diff in [1, 2,.., |w|] \+
    \\ for n in [1,2,...,|w|] \+
    \\ ($n$, $m$) = ($n$, $n + \text{diff}$)
    \\ // Make sure $n$ and $m$ are in bounds
    \\ // Calculate the value in $O(1)$ time.
\end{algo}

Now we need to actually figure out the lowest cost. First, note that our $min(l)$ function is the standard linear time min function taking a list as input. 
\\ Let's define a function $LowestCost(i)$ that is the lowest cost palindrome decomposition for $w$ that starts from the $i$th letter, ie $LowestCost(1)$ is the final output we want in this problem, and $LowestCost(|w|)$ is the lowest cost of decomposing the last letter of $w$ into a palindrome, namely $cost(1)$.
\[
    \emph{LowestCost}(i) =
	\begin{cases}
		0 & \text{if $i > |w|$}
\\[0.5ex]
        min(cost(m - n + 1) + LowestCost(m + 1) \mid (n, m) \in \{(nn, mm) \mid nn = i 
        \\ \qquad{} \wedge \emph{IsPalindrome}(nn, mm) = \textsc{True}\}) & \text{if $n = m - 1$}
\\[0.5ex]
	\end{cases}
\]
Note that this can run in $O(n)$ time because the linear search in min only has to cover $O(n)$ items, not all $O(n^2)$, since we know the value of $n$ \textit{a priori}.
\\ We can memoize this using a 1-d array. Since each value in our array only depends on the next value, we can start from $|w|$ and go down to 1 to fill in our array, each calculation taking linear time. Since we are doing this $O(n)$ time operation $O(n)$ times, the total time for this part is $O(n^2)$. As stated before, once we finsh filling our array, the answer we want is just in the first element. Thus, we can get the answer to the whole problem in $O(n^2)$ time.
\end{solution}
\end{document}
