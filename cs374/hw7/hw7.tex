% ---------
%  Compile with "pdflatex hw0".
% --------
%!TEX TS-program = pdflatex
%!TEX encoding = UTF-8 Unicode

\documentclass[11pt]{article}
\usepackage{jeffe,handout,graphicx}
\usepackage[utf8]{inputenc}		% Allow some non-ASCII Unicode in source
\usepackage{amsmath}
\usepackage[makeroom]{cancel}

%  Redefine suits
\usepackage{pifont}
\def\Spade{\text{\ding{171}}}
\def\Heart{\text{\textcolor{Red}{\ding{170}}}}
\def\Diamond{\text{\textcolor{Red}{\ding{169}}}}
\def\Club{\text{\ding{168}}}

\def\Cdot{\mathbin{\text{\normalfont \textbullet}}}
\def\Sym#1{\textbf{\texttt{\color{BrickRed}#1}}}

\newcommand{\IsSinL}{\text{IsStringInL}}
\newcommand{\IsSinlang}[1]{\text{IsStringIn}L_{#1}}
\newcommand{\cost}{\text{cost}}



% =====================================================
%   Define common stuff for solution headers
% =====================================================
\Class{CS/ECE 374}
\Semester{Fall 2018}
\Authors{1}
\AuthorOne{Will Koster}{jameswk2@illinois.edu}
%\Section{}

% =====================================================
\begin{document}

% ---------------------------------------------------------


\HomeworkHeader{6}{1}	% homework number, problem number

\begin{quote}
The McKing chain wants to open several restaurants along Red
  street in Shampoo-Banana. The possible locations are at $L_1,L_2,
  \ldots, L_n$ where $L_i$ is at distance $m_i$ meters from the start
  of Red street. Assume that the street is a straight line and the
  locations are in increasing order of distance from the starting
  point (thus $0 \leq m_1 < m_2 < \ldots < m_n$). McKing has collected
  some data indicating that opening a restaurant at location $L_i$
  will yield a profit of $p_i$ independent of where the other
  restaurants are located. However, the city of Shampoo-Banana has a
  zoning law which requires that any two McKing locations should be
  $D$ or more meters apart. {\em In addition McKing does not want to
    open more than $k$ restaurants due to budget constraints.}
  Describe an algorithm that McKing can use to figure out the maximum
  profit it can obtain by opening restaurants while satisfying the
  city's zoning law and the constraint of opening at most $k$
  restaurants.  Your algorithm should use only $O(n)$ space and you
  should not assume that $k$ is a constant.
\end{quote}
\hrule



\begin{solution}
    First, let's define a \emph{next}(x) function that gives you the smallest $i$ such that $L_i$ is greater than x.
    Now let's define our recurrence. This function $MP(i, k)$ gives you the Maximum Profit you can get given locations $L_j$ such that $j \geq i$ and $k$ restaurants left in your budget. The idea is that, at any location, you can either put a restaurant there or not. This function calculates both options and returns the optimal one. $MP(1, k)$ (where $k$ is your original k) gives your final answer.
\[
    MP(i, 0) = 0
\]
\[
    MP(i > n, k) = 0
    \]
\[
    MP(i, k) = max(MP(\emph{next}(i + d), k - 1), MP(i + 1, k))
    \]

    Note that our BinarySearch function returns the leftmost element that is greater than or equal to the "needle" input. Also, we are memoizing using 2 1-d arrays.
    \begin{algo}
        MaxMoney(L[1..n], k, d) \+
        L[n + 1] = $\infty$
        \\ k = [0] * n // k is the k we're at 
        \\ km = [0] * n // km is our k - 1 
        \\ k[n + 1] = km[n + 1] = 0
        \\ currK = 1 // we'll get 0 for everything if k = 0, so we start with 
        \\ // our k tracking k = 1
        \\ while currK <= k \+
        \\ for i $\gets$ n $\ldots$ 1 \+
        \\ next = BinarySearch($L$, $L_i + d$)
        \\ k[i] = max(km[next], k[i + 1]) \-
        \\ currK++
        \\ km = k \-
        \\ return km[0]
    \end{algo}
    This runs in $O(kn)$ time, taking $O(n)$ space.
\end{solution}

\HomeworkHeader{6}{2}	% homework number, problem number

\begin{quote}
Let $X = x_1,x_2,\ldots,x_r$, $Y = y_1,y_2,\ldots,y_s$ and $Z =
  z_1,z_2,\ldots,z_t$ be three sequences. A common {\em supersequence}
  of $X$, $Y$ and $Z$ is another sequence $W$ such that $X$, $Y$ and $Z$
  are subsequences of $W$. Suppose $X = a,b,d,c$ and $Y = b,a,b,e,d$ and $Z =
  b, e, d, c$. A simple common supersequence of $X$, $Y$ and $Z$ is
  the concatenation of $X$, $Y$ and $Z$ which is
  $a,b,d,c,b,a,b,e,d,b,e,d,c$ and has length $13$. A shorter one is
  $b, a, b, e, d, c$ which has length $6$.  Describe an efficient
  algorithm to compute the {\em length} of the shortest common
  supersequence of three given sequences $X$, $Y$ and $Z$. You may want to
  first solve the two sequence problem to get you strated.
\end{quote}
\hrule



\begin{solution}
    Our recurrence finds the shortest common superstring of x[i..n], y[j..n], and z[j..n]. Also, let x[n + 1] = $\xi$, y[n + 1] = $\upsilon$, and z[n + 1] = $\zeta$ and assume our three original strings were using only the Latin alphabet. The idea behind this is that, if you run out of letters in any string, the other two strings will proceed as if they were alone because the Greek letters will never be the same as any other letter in any of the strings. Calling $MS(1,1,1)$ will yield the answer to this problem.
    \[
        \emph{MS}(i, j, k) =
        \begin{cases}
            \infty & \text{if i > n + 1 or j > n + 1 or k > n + 1}
            \\[0.5ex]
            1 + min(MS(i + 1, j + 1, k), MS(i, j, k + 1)) & \text{if x[i] = y[j]}
            \\[0.5ex]
            1 + min(MS(i + 1, j, k + 1), MS(i, j + 1, k)) & \text{if x[i] = z[k]}
            \\[0.5ex]
            1 + min(MS(i, j + 1, k + 1), MS(i + 1, j, k)) & \text{if y[j] = z[k]}
            \\[0.5ex]
            1 + MS(i + 1, j + 1, k + 1) & \text{if x[i] = y[j] = z[k]}
            \\[0.5ex]
            1 + min(MS(i + 1, j, k), MS(i, j + 1, k), MS(i, j, k + 1)) & \text{if x[i], y[j], z[k] distinct}
            \\[0.5ex]
        \end{cases}
    \]
    The actual code looks like this. We are going to memoize with a 3-d array.
    \begin{algo}
        MS(x[1..n], y[1..n], z[1..n]) \+
        \\ memo = $(n + 2) \times (n + 2) \times (n + 2)$ 3-d array initialized to $\infty$
        \\ // Having a coordinate be n means you're on the last letter, n + 1 means that string is done
        \\ // (but the other two might still be going), and n + 2 means you've overshot and have to go back.
        \\ x[n + 1] = $\xi$
        \\ y[n + 1] = $\upsilon$
        \\ z[n + 1] = $\zeta$
        \\ for i $\gets$ n $\ldots$ 1 \+
        \\ for j $\gets$ n $\ldots$ 1 \+
        \\ for k $\gets$ n $\ldots$ 1 \+
        \\ if i = j = k = n + 1 \+
        \\ // This is the end; if we get here, we're done
        \\ memo[i,j,k] = 0
        \\ continue \-
        \\ xx = x[i]
        \\ yy = y[j]
        \\ zz = z[k]
        \\ if xx = yy then mem[i,j,k] = 1 + min(mem[i + 1, j + 1, k], mem[i, j, k + 1])

        \\ else if xx = zz then mem[i,j,k] = 1 + min(mem[i + 1, j, k + 1], mem[i, j + 1, k])

        \\ else if zz = yy then mem[i,j,k] = 1 + min(mem[i + 1, j, k], mem[i, j + 1, k + 1])

        \\ else if xx = yy = zz then mem[i,j,k] = 1 + mem[i + 1, j + 1, k + 1]

        \\ else mem[i,j,k] = 1 + min(mem[i + 1, j, k], mem[i, j + 1, k], mem[i, j, k + 1]) \-\-\-
        \\ return mem[0,0,0]
    \end{algo}
    This will take $O(n^3)$ time.
\end{solution}

\HomeworkHeader{6}{3}	% homework number, problem number

\begin{quote}
Given a graph $G=(V,E)$ a matching is a subset of edges in $G$
  that do not \emph{intersect}. More formally $M \subseteq E$ is a
  matching if every vertex $v \in V$ is incident to at most one edge
  in $M$. Matchings are of fundamental importance in combinatorial
  optimization and have many applications. Given $G$ and non-negative
  weights $w(e), e \in E$ on the edges one can find the maximum weight
  matching in a graph in polynomial time but the algorithm requires
  advanced machinery and is beyond the scope of this course. However,
  finding the maximum weight matching in a tree is easier via dynamic
  programming. 

  \begin{itemize}
  \item Solve the previous part even though it is not required to be
    submitted for grading. It will help you think about this part.
    \begin{enumerate}
    \item Given a tree $T=(V,E)$ describe an efficient algorithm to
      \emph{count} the number of distinct matchings in $T$. Two matchings
      $M_1$ and $M_2$ are distinct if they are not identical as sets of edges.
      Unlike the maximum weight matching problem, the problem of couting
      matchings is known to be hard in general graphs, but trees are
      easier. 
    \item Write a recurrence for the exact number of matchings in a
      path on $n$ nodes. For the base case of a single node tree
      assume that the answer is $1$ since an empty set is also a valid
      matching. Would the answer for a path with $n=500$ fit in a
      $64$-bit integer word? Briefly justify your answer.
    \item How would you implement your counting algorithm from part
      (a), more carefully, to run on a $64$ bit machine?  Accounting
      for this more careful implementation, what is the running time of
      your algorithm? 
    \end{enumerate}
  \end{itemize}
\end{quote}
\hrule



\begin{solution}
    \begin{enumerate}
        \item Our algorithm will go through the tree and set the \texttt{count} field on every node to the count of the matchings in that node's subtree. It will start at the leaves and go up. Our final answer will be the \texttt{count} field of the root. Call $CountMatchings$(G) to get that final answer. We will memoize using the same tree that we were given.
            \begin{algo}
                CountMatchings(G) \+
                \\ Pick some node r and create a tree T rooted at r
                \\ countMatchings(r)
                \\ return r.count \-
                \\ 
                \\ countMatchings(r) \+
                \\ if isLeaf(r) \+
                \\ r.count = 1
                \\ return \-
                \\ // Make sure that all my children have been done
                \\ for c in children(r) \+
                \\ countMatchings(c) \-
                \\ 
                \\ ct = 0
                \\ for c in children(r) + [null] \+
                \\ // Each each iteration of this loop signifies one edge from r being selected, 
                \\ // and null means none of the edges were selected.
                \\ prod = 1 
                \\ for ch in children(r) \+
                \\ if ch == c then continue
                \\ prod *= ch.count \-
                \\ // Note that children(c) are the grandchildren of r by c. Since the edge from r to c is selected, none of
                \\ // c's child edges can be selected. Also, assume that children(null) gives an empty list.
                \\ for ch in children(c) \+
                \\ prod *= ch.count \-
                \\ ct += prod \-
                \\ r.count = ct \-
            \end{algo}
            Finding the value of a single node will take at worse $O(n^2)$ time, so finding the value of all the nodes will take $O(n^3)$ time.
        \item 
            \[
                T(1) = 1
            \]
            \[
            T(2) = 2
            \]
            \[
            T(n) = T(n - 1) + T(n - 2)
            \]
            This is the Fibonacci sequence. WolframAlpha tells me that the 500th Fibonacci number is 105 decimal digits long, which is quite a lot more than 64 binary digits.
        \item On a real computer, computing a number that big would require something like a BigInteger, but operations on those do not take constant time. Specifically, addition takes linear time in the length of the number, and multiplication takes $O(n^{\log_2{3}})$ using Karatsuba's algorithm. Previously, we assumed that each of those operations took constant time, so we came up with $O(n^3)$ as the amount of work done for each node. We can ignore addition since it is faster than multiplication. In the worst case, each subtree has a path of length $O(n)$ below it, meaning that it's value is something in the neighborhood of $F_n = O(\phi^n)$, meaning it has $O(n)$ digits. Therefore, multiplying the cost of each multiplication by the number of multiplications, we get a new run time of $O(n^{3\log_2{3}})$.
    \end{enumerate}
\end{solution}
\end{document}
