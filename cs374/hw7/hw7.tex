% ---------
%  Compile with "pdflatex hw0".
% --------
%!TEX TS-program = pdflatex
%!TEX encoding = UTF-8 Unicode

\documentclass[11pt]{article}
\usepackage{jeffe,handout,graphicx}
\usepackage[utf8]{inputenc}		% Allow some non-ASCII Unicode in source
\usepackage{amsmath}
\usepackage[makeroom]{cancel}

%  Redefine suits
\usepackage{pifont}
\def\Spade{\text{\ding{171}}}
\def\Heart{\text{\textcolor{Red}{\ding{170}}}}
\def\Diamond{\text{\textcolor{Red}{\ding{169}}}}
\def\Club{\text{\ding{168}}}

\def\Cdot{\mathbin{\text{\normalfont \textbullet}}}
\def\Sym#1{\textbf{\texttt{\color{BrickRed}#1}}}


% =====================================================
%   Define common stuff for solution headers
% =====================================================
\Class{CS/ECE 374}
\Semester{Fall 2018}
\Authors{1}
\AuthorOne{Will Koster}{jameswk2@illinois.edu}
%\Section{}

% =====================================================
\begin{document}

% ---------------------------------------------------------


\HomeworkHeader{7}{1}	% homework number, problem number

\begin{quote}
Let $G=(V,E)$ be \emph{directed} graph. A subset of vertices are colored
  red and a subset are colored blue and the rest are not colored.  Let
  $R \subset V$ be the set of red vertices and $B \subset V$ be the set
  of blue vertices.
  \begin{itemize}
  \item Describe an efficient algorithm that given $G$ and
  two nodes $s,t \in V$ checks whether there is an $s$-$t$ path in $G$
  that contains at most one red vertex and at most one blue
  vertex. For simplicity assume that $s,t$ do not have colors. Ideally
  your algorithm should run in $O(n+m)$ time where $n = |V|$ and $m = |E|$.
  Do not try to invent a new algorithm. Come up with a way to create
  a new graph $G'$ and use a standard algorithm on $G'$.
\item Here is a small variation where edges are colored instead of
  vertices.  Some of the edges in $G$ are colored red and some are
  colored blue and the rest are not colored. Let $R \subset E$ be the
  red edges and $B \subset E$ be the blue edges. Describe an efficient
  algorithm that given $G$ and two nodes $s,t$ checks whether there is
  an $s$-$t$ path that contains at most one red edge and at most one
  blue edge. Reduce the problem to the one in the previous part.
  \end{itemize}
\end{quote}
\hrule



\begin{solution}
    \begin{itemize}
        \item Our new graph $G'$ will look like 4 copies of $G$ with slight changes. One will be for when we have seen neither a red nor a blue vertex (RB), one for when we have seen a red vertex (B), one for when we have seen a blue vertex (B), and one for when we have seen a red and a blue vertex (N). Once you see a colored vertex, you will move to a graph that has no vertices of that color. If you can find a way from the starting vertex $(s, RB)$ to any vertex of the form $(t, p)$ for $t$ as your ending vertex, there is a path in the orignal graph that touches at most one red and one blue vertex.
            \[
                G' = (V', E')
            \]
            \[
                V' = V \times \{RB, B, R, N\} \setminus \{(u, R) \mid u \in V, \text{isBlue}(u)\} \setminus \{(u, B) \mid u \in V, \text{isRed}(u)\} \setminus \{(u, n) \mid u \in V, \text{isRed}(u) \lor \text{isBlue}(u)\}
            \]
            \[
                NC = \{u \mid u \in V, \lnot(\text{isRed}(u) \lor \text{isBlue}(u))\}
            \]
            \[
                C = \{u \mid u \in V, u \not\in NC\}
            \]
            \[ 
                \text{NormalEdges} = \{((u, m), (v, m) \mid u \in C, (u, m) \in V', (v, m) \in V', (u, v) \in E\}
            \]
            \[
                \text{NextSubgraph}(BR, v) = R \text{ if isBlue}(v)
            \]
            \[
                \text{NextSubgraph}(BR, v) = B \text{ if isRed}(v)
            \]
            \[
                \text{NextSubgraph}(R, v) = N \text{ if isRed}(v)
            \]
            \[
                \text{NextSubgraph}(B, v) = N \text{ if isBlue}(v)
            \]
            \[
                \text{FromColored} = \{((u, w), (v, \text{NextSubgraph}(w, u)) \mid u \in C, (u, t) \in V', (v, \text{NextSubgraph}(w, u)) \in V', (u, v) \in E\}
            \]
            Once you have this graph, it's just a search problem. Use whatever-first search starting from $(s, RB)$ searching for any any vertex of the form $(t, m)$ for $t$ as your ending vertex. We know whatever-first search takes $O(|V| + |E|)$ time in general, and we know that our $|V'| = O(|V|)$, since it's just a set multiplication with a finite set. $|E'| = O(|E|)$ since we are repeating each edge a finite number of times. This means that search will run in $O(|V| + |E|)$ time in this specific case too.
        \item We can transform this graph into the graph from above by just replacing every colored edge $(u, v)$ with a new vertex $w$ of the same color and two regular edges, $(u, w)$ and $(w, v)$. Say we make a list of all the colored edges $L_{1..n}$.
            \[
                V' = V \cup \{\text{new vertices $v_i$ for each edge $L_i$ $\mid$ $v_i$ has the same color as $L_i$}\}
            \]
            \[
                E' = E \cup \bigcup^L_{L_i = (u, v)}{\{(u, v_i), (v_i, v)\}}\setminus L
            \]
            Use the same algorithm as above. $|V'| = O(|V| + |E|)$ and $|E'| = O(|E|)$, so the search will still run in $O(|V| + |E|)$ time.
    \end{itemize}
\end{solution}

\HomeworkHeader{7}{2}	% homework number, problem number

\begin{quote}
Let $G=(V,E)$ be a directed graph.
  \begin{itemize}
  \item Describe a linear-time algorithm that outputs all the nodes in
    $G$ that are contained in some cycle. More formally you want to
    output
    $$S = \{ v \in V \mid \text{there is some cycle in $G$ that
      contains v}\}.$$
  \item Describe a linear time algorithm to check whether there is a
    node $v \in V$ such that $v$ can reach every node in $V$. First
    solve the problem when $G$ is a DAG and then generalize it via the
    meta-graph construction.
  \end{itemize}
  No proofs necessary but your algorithm should be clear. Use known
  algorithms as black boxes rather. In particular the linear-time algorithm to
  compute the meta-graph is useful here.
\end{quote}
\hrule



\begin{solution}
    \begin{itemize}
        \item First, use the linear-time SCC-finding algorithm we learned in lecture to find all the strongly-connected components. Then just make a list of every vertex in a non-singleton SCC. All those vertices are in cycles. There are obviously no more than $|V|$ strongly-connected components, so that part will take linear time, so we know the whole algorithm will take linear time.
        \item If you have a DAG with 1 source, which we will call the root, then you can get to any node from that root, since at that point we will just have a regular tree. If you have a DAG with 2 or more sources, we know that neither source can get to any of the other sources. We know DAGs can't have zero sources. Therefore, for a DAG, there exists a vertex $v$ that can reach all other vertices iff there is exactly 1 sink. Finding all the sinks is trivial with the adjacency list representation, and if we make the reverse graph in linear time, then the sources in our original graph will be the sinks in our new graph. Therefore, this can be done in linear time. If we have some general graph, we know we can make the SCC meta-graph in linear time, and the above will apply to that meta-graph. Therefore, to find whether some graph $G$ has a vertex $v$ that can reach every other vertex in $G$, just find the SCC meta-graph and count the sources. If there is exactly 1, there is such a vertex, and if there are more than 1, there is no such vertex.
    \end{itemize}
\end{solution}

\HomeworkHeader{7}{3}	% homework number, problem number

\begin{quote}
 Given an \emph{undirected} connected graph $G=(V,E)$ an edge $(u,v)$ is
  called a cut edge or a bridge if removing it from $G$ results in
  two connected components (which means that $u$ is in one component
  and $v$ in the other). The goal in this problem is to design an efficient
  algorithm to find {\em all} the cut-edges of a graph.

  \begin{itemize}
  \item What are the cut-edges in the graph shown in the figure?

  \item Given $G$ and edge $e=(u,v)$ describe a linear-time algorithm
    that checks whether $e$ is a cut-edge or not. What is the running time
    to find all cut-edges by trying your algorithm for each edge? No proofs
    necessary for this part.
  \item Consider {\em any} spanning tree $T$ for $G$. Prove that every
    cut-edge must belong to $T$. Conclude that there can be at most $(n-1)$
    cut-edges in a given graph. How does this information improve the
    algorithm to find all cut-edges from the one in the previous step?
  \item Suppose $T$ is any spanning tree of $G$. Root it at some
    arbitrary node.  Prove that an edge $(u,v)$ in $T$ where $u$ is
    the parent of $v$ is a cut-edge iff there is no edge in $G$, other
    than $(u,v)$, with one end point in $T_v$ (sub-tree of $T$ rooted
    at $v$) and one end point outside $T_v$.
  \item Now consider the DFS tree $T$.  Use the property in the
    preceding part to design a linear-time algorithm that outputs all
    the cut-edges of $G$. What additional information can you maintain
    while running DFS? Recall that there are no cross-edges in a DFS
    tree $T$. You don't have to prove the correctness of
    the algorithm.
  \end{itemize}
\end{quote}
\hrule



\begin{solution}
    \begin{itemize}
        \item Cut edges are $(h, l)$, $(f, j)$, and $(g, e)$.
        \item We can just create a new graph $G' = (V, E - \{e\})$ and use some search only once (ie, pick exactly one random starting vertex and search from there. When you run out of unvisited neighbors, stop.) If every vertex is marked after that, then your edge was not a cut edge, otherwise it was. This takes linear time, so finding all the cut edges would take $O(|E|(|E| + |V|))$ time.
        \item Let's prove that, for any spanning tree $T$ on $G$, all the cut edges of $G$ are in $T$. 
            \\ Assume the contratrary. That is, assume that there is some cut edge $e$ that is not in $T$. 
            \\ By the definition of a spanning tree, $T$ is connected. By our definition of $T$, $e \not \in T$. Therefore, if we have some new graph $G' = (V, E - \{e\})$, $T$ will be a spanning tree of $G'$ too.
            \\ By the definition of a cut edge, $G'$ is disconnected. But it has a spanning tree $T$, so it must be connected. This is a contradiction.
            \\ Therefore, we conclude our original proposition. QED
        \item By finding a spanning tree (which we can do in linear time with DFS), we have narrowed down the number of possible cut edges to $|V| - 1$, which is no greater than $|E|$ in a connected graph. Therefore, we can run the na\"{i}ve algorithm from above faster by only trying the edges in our spanning tree. This would run in $O(|V|(|V| + |E|))$.
        \item Since we are proving an iff relationship, we're just going to use a 2-way proof. First I'll prove that if $(u, v)$ is a cut-edge, then there is no other edge in $G$ with one end outside $T_v$ and the other inside. Then I'll prove the converse.
            \\ \\ Say we have some edge $e = (u, v)$ in a spanning tree $T$ on $G$ such that $u$ is the parent of $v$ in $T$. Say we know that $e$ is a cut edge of $G$. Let's prove that there is no other edge in $G$ that starts outside the subtree of $T$ rooted at $v$ $T_v$ and ends inside $T_v$.
            \\ Let's assume the contrary and say that we have some edge $f = (a, b)$ with one endpoint (say it's $a$ without loss of generality) inside $T_v$ and one outside (say it's $b$). By the defnition of a cut edge, there is a graph $G' = (V, E - \{e\})$ that is disconnected, and in $G'$ $u$ and $v$ are in different connected components. Since $a$ is in $T_v$, we know that $a$ reaches $v$. We know that $a$ is outside $T_v$ but it must be connected to $v$ (and therefore not connected to $u$) in $G'$. Therefore, $a$ is not one of $v$'s children. It cannot be its parent because we know $u$ was $v$'s parent, and $a$ can't be $v$'s sibling because they would have to be connected through a parent. We have a contradiction. Therefore, there can be no such edge $f$.
            \\ \\ Now let's prove the converse. We have some edge $e = (u, v)$, where $u$ is the parent of $v$. We know that there is no edge in $G$ other than $e$ that goes between some vertex inside $T_v$ and a vertex outside it. Let's prove that $e$ is a cut edge.
            \\ We are given that the only edge between any vertex in $T_v$ and the rest of the graph is $e$. Therefore, we know that the path between any vertex in $T_v$ and any vertex not in $T_v$ must contain $e$. If we remove $e$ from the graph, we know that there can be no such path. Therefore, if nothing in $T_v$ can reach anything outside $T_v$ in $G' = (V, E - \{e\})$, we know that $G'$ is disconnected. Since removing $e$ gives us a disconnected graph, we know by the definition that $e$ is a cut edge.
            \\ Since we proven the implication both ways, we have that the two-way implication holds. 
        \item
            This algorithm is from https://www.cc.gatech.edu/~rpeng/CS3510\_F16/notes/Sep28DFS.pdf
            \begin{algo}
                CutEdges($G = (V, E)$) \+
                \\ Run DFS(G), marking on each vertex the time it was explored
                \\ and returning the DFS tree as $T = (V', E')$
                \\ and the back edges $B$.
                \\ for all $v$ in $V$ set v.farthest = -1
                \\ edges = $\emptyset$
                \\ FindFarthestBackedge(T.root)
                \\ for $(u, v)$ in $E'$ \+
                \\ if u.farthest $\geq$ v.time then add $(u, v)$ to edges \-
                \\ return edges \-
                \\ FindFarthestBackedge($v$) \+
                \\ if v.farthest $\neq$ -1 then return v.farthest
                \\ minChild = min(\{FindFarthestBackedge(c) $\mid$ c is a child of v\})
                \\ myMinBackedge = min(\{b.time $\mid$ $(v, b)$ where $(v, b)$ is a back edge\})
                \\ v.farthest = min(minChild, myMinBackedge)
                \\ return v.farthest
            \end{algo}
            Since each node is only touched a constant number of times to calculate the farthest-reaching back edge, that part runs in linear time, so the whole algorithm runs in $O(|V| + |E|)$ time. Calling $CutEdges(G)$ will give you the answer.
    \end{itemize}
\end{solution}
\end{document}
