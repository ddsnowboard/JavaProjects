% ---------
%  Compile with "pdflatex hw0".
% --------
%!TEX TS-program = pdflatex
%!TEX encoding = UTF-8 Unicode

\documentclass[11pt]{article}
\usepackage{jeffe,handout,graphicx}
\usepackage[utf8]{inputenc}		% Allow some non-ASCII Unicode in source
\usepackage{amsmath}
\usepackage[makeroom]{cancel}

%  Redefine suits
\usepackage{pifont}
\def\Spade{\text{\ding{171}}}
\def\Heart{\text{\textcolor{Red}{\ding{170}}}}
\def\Diamond{\text{\textcolor{Red}{\ding{169}}}}
\def\Club{\text{\ding{168}}}

\def\Cdot{\mathbin{\text{\normalfont \textbullet}}}
\def\Sym#1{\textbf{\texttt{\color{BrickRed}#1}}}

\newcommand{\IsSinL}{\text{IsStringInL}}
\newcommand{\IsSinlang}[1]{\text{IsStringIn}L_{#1}}
\newcommand{\cost}{\text{cost}}



% =====================================================
%   Define common stuff for solution headers
% =====================================================
\Class{CS/ECE 374}
\Semester{Fall 2018}
\Authors{1}
\AuthorOne{Will Koster}{jameswk2@illinois.edu}
%\Section{}

% =====================================================
\begin{document}

% ---------------------------------------------------------


\HomeworkHeader{7}{1}	% homework number, problem number

\begin{quote}
Let $G=(V,E)$ be \emph{directed} graph. A subset of vertices are colored
  red and a subset are colored blue and the rest are not colored.  Let
  $R \subset V$ be the set of red vertices and $B \subset V$ be the set
  of blue vertices.
  \begin{itemize}
  \item Describe an efficient algorithm that given $G$ and
  two nodes $s,t \in V$ checks whether there is an $s$-$t$ path in $G$
  that contains at most one red vertex and at most one blue
  vertex. For simplicity assume that $s,t$ do not have colors. Ideally
  your algorithm should run in $O(n+m)$ time where $n = |V|$ and $m = |E|$.
  Do not try to invent a new algorithm. Come up with a way to create
  a new graph $G'$ and use a standard algorithm on $G'$.
\item Here is a small variation where edges are colored instead of
  vertices.  Some of the edges in $G$ are colored red and some are
  colored blue and the rest are not colored. Let $R \subset E$ be the
  red edges and $B \subset E$ be the blue edges. Describe an efficient
  algorithm that given $G$ and two nodes $s,t$ checks whether there is
  an $s$-$t$ path that contains at most one red edge and at most one
  blue edge. Reduce the problem to the one in the previous part.
  \end{itemize}
\end{quote}
\hrule



\begin{solution}
    cats
\end{solution}

\HomeworkHeader{7}{2}	% homework number, problem number

\begin{quote}
Let $G=(V,E)$ be a directed graph.
  \begin{itemize}
  \item Describe a linear-time algorithm that outputs all the nodes in
    $G$ that are contained in some cycle. More formally you want to
    output
    $$S = \{ v \in V \mid \text{there is some cycle in $G$ that
      contains v}\}.$$
  \item Describe a linear time algorithm to check whether there is a
    node $v \in V$ such that $v$ can reach every node in $V$. First
    solve the problem when $G$ is a DAG and then generalize it via the
    meta-graph construction.
  \end{itemize}
  No proofs necessary but your algorithm should be clear. Use known
  algorithms as black boxes rather. In particular the linear-time algorithm to
  compute the meta-graph is useful here.
\end{quote}
\hrule



\begin{solution}
    dogs
\end{solution}

\HomeworkHeader{7}{3}	% homework number, problem number

\begin{quote}
 Given an \emph{undirected} connected graph $G=(V,E)$ an edge $(u,v)$ is
  called a cut edge or a bridge if removing it from $G$ results in
  two connected components (which means that $u$ is in one component
  and $v$ in the other). The goal in this problem is to design an efficient
  algorithm to find {\em all} the cut-edges of a graph.

  \begin{itemize}
  \item What are the cut-edges in the graph shown in the figure?

  \item Given $G$ and edge $e=(u,v)$ describe a linear-time algorithm
    that checks whether $e$ is a cut-edge or not. What is the running time
    to find all cut-edges by trying your algorithm for each edge? No proofs
    necessary for this part.
  \item Consider {\em any} spanning tree $T$ for $G$. Prove that every
    cut-edge must belong to $T$. Conclude that there can be at most $(n-1)$
    cut-edges in a given graph. How does this information improve the
    algorithm to find all cut-edges from the one in the previous step?
  \item Suppose $T$ is any spanning tree of $G$. Root it at some
    arbitrary node.  Prove that an edge $(u,v)$ in $T$ where $u$ is
    the parent of $v$ is a cut-edge iff there is no edge in $G$, other
    than $(u,v)$, with one end point in $T_v$ (sub-tree of $T$ rooted
    at $v$) and one end point outside $T_v$.
  \item Now consider the DFS tree $T$.  Use the property in the
    preceding part to design a linear-time algorithm that outputs all
    the cut-edges of $G$. What additional information can you maintain
    while running DFS? Recall that there are no cross-edges in a DFS
    tree $T$. You don't have to prove the correctness of
    the algorithm.
  \end{itemize}
\end{quote}
\hrule



\begin{solution}
    wolves
\end{solution}
\end{document}
